\chapter{Conclusion}

In this thesis a base for random first order logic formula generator has been presented. The generator can be easily extended for user specific needs. For example user can add arbitrary rule (by inheriting one of generators) that elements must follow during generation, more output formats can be added.
Presented solution can be extended to follow constraints related to any other \gls{FOL} element like limited number of variables, functors and so on.
The biggest challenge in described solution is randomizing solutions from constraint solver. Solving user constrains is problem of integer programming (NP problem) that is why enumerating them all is not an option. Instead random number of solutions is skipped to preserve at least pseudo-randomness.

Alternative approach to using \gls{SMT} solver would be to take locally optimal decisions and introduce some randomness when taking those decisions. This approach would create arguably "more" random result but introduces number of drawbacks:

\begin{itemize}
  \item taking many local decisions may not be faster than taking one global - that requires in depth analysis,
  \item this is non deterministic approach - some possible variants of formula might never/rarely be reached, one can argue if it is even possible to create such algorithm
  \item adding new constraint would be much more complicated
\end{itemize}

The future development of presented generator would allow to generate formulas in different normal forms, more readable suggestions for constrains errors and ability to auto correct user constraints errors.
