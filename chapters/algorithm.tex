\chapter{Generator parameters and algorithm}

When talking about formula we would often describe formula by $n$ number of elements, eg. formula has 4 predicates. In sense of implementation more specific vocabulary is needed. From now on when talking about $n$ occurrences of element, mathematical sense is assumed (usually it is element name and arity). In implementation user will be asked to provide \textit{signature} of element. Signature is differentiated by argument types that is holds. This is important thus name of signature is often irrelevant and can be modified later. Element \textit{instance} is every occurrence of in formula. See \ref{tab:signatureComparison}

\begin{table}
  \centering
  \footnotesize
  \begin{tabularx}{\textwidth}{|c|X|X|}
    \hline
    FOL element & Unique element in mathematical sense & Unique element as signature \\
    \hline
    Variable & by name and scope $V1$, $V2, \dots$ & there is only one signature for variable \\  
    \hline
    Functor & by name and arity $f/0$, $f/1, \dots$ & by types of arguments $f(Variable)$, $f(functor), \dots$ \\
    \hline
    Predicate & by name and arity $p/0$, $p/1$, \dots & by types of arguments $p(Variable)$, $p(functor), \dots$ \\
    \hline
    Atom & by connective and operands $p/0$, $p/1 = V, \dots$ & by connective and operands signatures $p(Variable)$, $p(Variable) = p(Functor), \dots$ \\
    \hline
  \end{tabularx}
  \caption{Comparison of how elements of FOL in sense of math and signature. Elements not included here simply does not make have signature}
  \label{tab:signatureComparison}
\end{table}

\section{CNF Generator parameters}

User defines generator in 3 steps.
\begin{enumerate}
  \item \textbf{Allowed first order logic signatures}

    In this step user defines set of allowed \gls{FOL} elements:
    \begin{itemize}
      \item set of allowed functor arities $a_f = \{0, 1, 2,\dots\}$
      \item maximum recursion depth $n$ for functors
      \item set of allowed predicate arities $a_p = \{0, 1, 2,\dots\}$
      \item set of atom allowed connectives, that is no connective or/and any subset of $AllowedConnectives = \{=, !=, \emptyset\}$
      % \item if negated literals are allowed
      \item set of allowed clause lengths $AllowedClausesLengths = \{1,2,\dots\}$
      \item number of clauses in formula $NumberOfClauses$
      \item number of literals in formula $NumberOfLiterals$
    \end{itemize}

  \item \textbf{How many instances of elements are allowed}

    In this step user defines what properties formula should have:
    \begin{itemize}
      \item formula contains from $c_{min}$ to $c_{max}$ clauses
      \item formula contains from $l_{min}$ to $l_{max}$ literals
    \end{itemize}

  \item \textbf{Post processing - names for term-like FOL elements and literal}

    In this step user defines:
    \begin{itemize}
      \item set of variable names $\{'v1','v2',\dots\}$
      \item set of functor names $\{'f1','f2',\dots\}$
      \item set of predicate names $\{'p1','p2',\dots\}$
      \item amount of literals to be negated
    \end{itemize}
\end{enumerate}

\section{Basic algorithm}

\begin{enumerate}
  \item Resolve user constraints \ref{sec:ResolveUserConstrains}
  \item Generate possible formula based on signatures \ref{chap:LogicInternalRepresentation}
  \item Post process - give unique names, add unary operators (negation)
  \item Export formula to file
\end{enumerate}

\section{Resolve user constraints}
\label{sec:ResolveUserConstrains}

User constrains are defined in input parameters as $AllowedClausesLengths$, $NumberOfClauses$, $NumberOfLiterals$. To generate random formula withing these constrains, number of clauses with appriopriate clause length must be computed.

CNF formula $F_{cnf}$ consists of unordered clauses $c1, c2, \dots$. 

\begin{align*}
	&F_{cnf}(x) = \{c1, c2, \dots\, cx\} \\
	\text{where }
		&x \text{ -- number of clauses in formula}
\end{align*}

If we group clauses by their length:
\begin{align*}
	&F_{cnf}(x) = \bigcup_{i=1}^c c_i \\
	\text{where }
		&c_i \text{ -- set of clauses with length i} 
\end{align*}

Number of literals in formula can be represented as:
\begin{align*}
	l(x) &= x_1|c_1| + x_2|c_2| + \dots + x_x|c_x| = \sum_{i=1}^{x} x_i |c_i| \\
	x &= x_1 + x_2 + \dots + x_n \\
	c_i &\in AllowedClausesLen: \forall_{i \neq j} c_i \neq c_j  \\
	\text{where }
		&x \text{ -- number of clauses in formula} \\ 
		&l(x) \text{ -- number of literals in formula} \\ 
		&|c_i| \text{ -- number of clauses with length i} 
\end{align*}

So in the end:

\begin{align}
	l(x) &= \sum_{i=1}^{x} x_i |c_i| \\
	x &= \sum_i^x x_i \\
	l_{min} &< l(x) < l_{max} \\
	c_{min} &< x < c_{max} \\
	\text{where } 
		&x \text{ -- number of clauses in formula} \nonumber \\
		&l(x) \text{ -- number of literals in formula} \nonumber  \\
		&|c_i| \text{ -- number of clauses with length i} \nonumber
\end{align}

