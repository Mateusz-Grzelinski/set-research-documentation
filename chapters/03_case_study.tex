\chapter{Case study - generating random system properties}

One of the uses of random \gls{FOL} can be to generate random formula that represent system properties. That formula can be used as input for benchmark.

\section{Properties of computer systems}

System properties were first discussed in context of concurrency \cite{Lampert77} as a tool for formal verification multiprocess programs. One of first proposed properties were safety and liveness. These properties can apply to computer systems in general and be expressed in different formal systems.

\textbf{Liveness} \cite{Klimek99} is system property, that states, that something good will eventually happen.
Liveness formula guarantees that there is at least one case, where formula evaluates to true.

\textbf{Safety} \cite{Klimek99} is system property, that states, that something bad will never happens.
Safety formula must always be satisfied.

\subsection{Safety and liveness representation in logic systems}

In \gls{FOL} liveness and safety can be expressed as quantifiers, safety as universal quantifier and liveness as existential quantifier. Every \gls{FOL} can be converted to \gls{CNF}, so system properties can be also represented in \gls{CNF}, if needed. It can be done for example with otter algorithm \cite{McC-Otter-URL} or skolemisation.

In \gls{FOL} system properties could look like (first is example with quantifiers, followed by CNF equivalent)\footnote{removing quantiviers from first order logic can be automated with TPTP utility using tptp2X utility, option \mintinline{text}{-t clausify:quaife} \ref{sub:AdditionalToolsInTPTPLibrary} }:
\begin{itemize}
  \item Safety formula: $\forall_X p(X) \equiv p(A)$
  \item Safety formula: $\forall_X p(X, X) \equiv p(A, B)$
  \item Liveness formula: $\exists_X p(X) \lor \equiv p1(f1)$
  \item Liveness formula: $\exists_X p(X, X) \equiv p(f1, f2)$
\end{itemize}

The combinations of quantifiers (that correspond to combinations of liveness and safety properties) are not discussed in this thesis. For example (first is example with quantifiers, followed by CNF equivalent):
\begin{itemize}
\item $\exists_Y \forall_X p(X, Y) \equiv p(f1, A)$
\item $\forall_X \exists_Y p(X, Y) \equiv p(A,f1(A)) $
\end{itemize}

Example of real life problem expressed in safety and liveness formulas is collisionless lights on crossroad.

\noindent
Problem: cars want to drive through crossroad.

\noindent
Safety property: only one light can be green.

\noindent
Liveness property: every light should eventually turn green.

\section{Generating dataset}

The goal is to generate liveness and safety clauses, but do not mix them. In order to achieve that, class $PredicateGenerator$ needs to be modified to yield either formulas with all variables (safety) or formulas with 0-arity functor (liveness). To achieve that $PredicateGenerator$ was subclassed and used to cleate alternative version of $CNFFormulaGenerator$.

\begin{figure}[H]
\begin{centering}
  \includegraphics[width=\textwidth]{logic-formula-generator/fol/safety_liveness_predicate_generator.png}
  \caption{Subclassed PredicateGenerator mimics safety and liveness formulas}
\end{centering}
\end{figure}

In this study the impact of ratio of number of atoms to number of clauses will be presented.
Formulas with 1000 atoms and 100, 200, 300, 400, 500 clauses were generated, 50 for each combination. Number of atoms and number of clauses can vary within 5\%, although solver prefers smaller numbers in general. The rest of parameters for formulasa is shown in listing \ref{lis:CNFSafetyLivenesSnippet}.

50 random formulas were generated and were benchmarked against solvers Prover9 and SPASS. Results are shown in picture \ref{pic:benchmark_results}

\begin{listing}[ht]
  \caption{Snippet for generating dataset of safety and liveness formulas}
  \label{lis:CNFSafetyLivenesSnippet}
\begin{minted}{python}
gen = CNFSafetyLivenessGenerator(
    variable_names={f'V{i}' for i in range(10)},
    functor_names={f'f{i}' for i in range(20)}, functor_arity={0},
    functor_recursion_depth=0,
    predicate_names={f'p{i}' for i in range(20)}, predicate_arities={i for i in range(5)},
    atom_connectives={''},
    clause_lengths={i for i in range(2, 11)},
    number_of_clauses=IntegerRange.from_relative(number_of_clauses, threshold),
    number_of_literals=IntegerRange.from_relative(number_of_literals, threshold),
    literal_negation_chance=0.1,
)
\end{minted}
\end{listing}

\begin{listing}[ht]
  \caption{Example of generated formula (limited)}
\begin{tptpcode}
% ----------------------------------------------------------------------------
% File      : 0.p 
% Syntax    : Number of clauses     :   95 ( 95 non-Horn;   0 unit;   - RR)
%             Number of atoms       :  950 (  0 equality)
%             Maximal clause size   :   10 ( 10 average)
%             Number of predicates  :   20 (206 propositional; 0-4 arity)
%             Number of functors    :   20 (940 constant;   0 arity)
%             Number of variables   :  943 (327 singleton)
%             Maximal term depth    :    0 (  - average)
% 
% ----------------------------------------------------------------------------
cnf(name,axiom,p4(V6)|p10(V1, V4, V9)|~p7|p17|p1(V4, V3, V9)|p7|p17|p1(f11, f11, f5)|p7|~p0(f2)).
cnf(name,axiom,p2(V7, V9)|p4(f19)|p2(f8, f5)|p10(f7, f8, f8)|p12|p6(V2, V6)|p14(f8, f16, f9, f16)|p3(f14, f3, f14, f18)|p11(V3, V9)|p12).
...
\end{tptpcode}
\end{listing}

% All of formulas were UNSAT. The reason may be low number of available names in comparison to number of clauses.

\begin{figure}[ht]
\begin{centering}
  \includegraphics[width=\textwidth]{logic-formula-generator/dataset_analysis/execution_times.png}
  \caption{SPASS and Prover9 execution times on 100 random formulas. All of formulas were UNSAT}
  \label{pic:benchmark_results}
\end{centering}
\end{figure}

