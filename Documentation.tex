\documentclass[a4paper,12pt]{article}
\usepackage{amsmath}
%\usepackage{polish}
\usepackage[polish]{babel}
\usepackage[utf8]{inputenc}
\usepackage[T1]{fontenc}
\usepackage{graphicx}
\usepackage{anysize}
\usepackage{enumerate}
\usepackage{times}
\usepackage{plain}
\usepackage{caption}
\usepackage{graphicx}
\usepackage{setspace}
\usepackage{multirow}
\usepackage{datetime}
\usepackage{listings}
\usepackage[colorlinks=true]{hyperref}
\usepackage[automake, acronyms, toc, nopostdot, nonumberlist]{glossaries-extra}
\usepackage{indentfirst}
\hypersetup{
  colorlinks=true,
  linkcolor=blue,
  filecolor=magenta,
  urlcolor=cyan,
}
\urlstyle{same}
%\marginsize{left}{right}{top}{bottom}
\marginsize{2.5cm}{2.5cm}{2.5cm}{2.5cm}
\setlength{\parindent}{4em}
\setlength{\parskip}{1em}
\renewcommand{\baselinestretch}{2.0}

\lstset{basicstyle=\small\ttfamily,breaklines=true}
\lstset{framextopmargin=50pt,frame=bottomline}

\makeglossaries
\loadglsentries{glossaries}
\glsaddall

\begin{document}
\onehalfspacing
\begin{figure}[!htb]
  \centerline{\includegraphics[scale=0.8]{images/agh_logo.jpg}}
\end{figure}

\begin{center}
  \Huge{Studio projektowe 1\\}
  \Large{Benchmark solverów prover9 oraz SPASS\\ \large \textit \today \\}
  \vspace{3cm}
  \Large{	Autorzy:\\
  Mateusz Grzeliński\\
  Przemysław Michałek\\
  }
  \large{Wydział Elektrotechniki, Automatyki, Informatyki i Elektroniki}

  \newpage
\end{center}

\tableofcontents
\newpage

\section{Streszczenie}

Celem projektu jest zbadanie wydajności automatyczych metod dowodzenia twierdzeń Prover9 oraz SPASS. Na początku zostaje wygenerowna formuła \gls{SAT}, która zostaje rozwiązana przez badane provery. Badany jest czas wykonania, rezultat (czy SAT jest spełnialny), użycie pamięci RAM.
Generowana formuła SAT jest modufikowana ze względu na długość formuły, ilość zmiennych.

\section{Benchmark - strategia blackbox}

W stategii blackbox traktujemy provery jako czarny skrzynki - nie ingerujemy w wykonywany program, badamy tylko input oraz output. Dla każdego provera dostępne są:

\begin{itemize}
  \item czas wykonania
  \item użycie pamięci RAM
\end{itemize}

Zbadane zostaną statystyki proverów ze względu na następujące właściwości formuły SAT:

\begin{itemize}
  \item ilość zmiennych
  \item stosunek ilości formuł do ilości zmiennych
  \item czy formuła jest w postaci \gls{CNF}
\end{itemize}


\begin{figure}[!htb]
  \includegraphics[scale=0.4]{images/studio-projektowe1.png}
  \caption{Diagram komponentów systemu benchmarka}
\end{figure}

\newpage

\subsection{Prover9}

Prover9 jest to zautomatyzowane narzędzie udowadniające dla logiki pierwszego rzędu stworzone przez Williama McCune’a.

Prover9 dostępny jest jako plik wykonywalny, przyjmuje pliki w formacie \gls{LADR}. Ilość opcji dostępnych z lini komend jest minimanlna, jedyną ważną opcją z punktu widzenia benchmarka jest
\lstinline{-x - set(auto2).  (enhanced auto mode)}

Oficjalna strona internetowa \url{https://www.cs.unm.edu/~mccune/mace4/}


\lstinputlisting[caption={Przykład pliku wejściowego w składni LADR}]{listings/prover9_example.in}
\lstinputlisting[caption={Przykład wyjścia Provera9}, basicstyle=\footnotesize\ttfamily]{listings/prover9_example.out}
\subsection{SPASS}

SPASS Theorem Prover jest narzędziem do automatycznego dowodzenia twierdzeń, należących do rachunku predykatów pierwszego rzędu.

SPASS nie korzysta z żadnych bibliotek, dostępny jest jako plik wykonywalny. Akceptuje pliki w składni TPTP lub swojej własnej. SPASS udostępnia wiele opcji z poziomu lini komend. Z punktu widzenia benchmarka istotnymi są:

\begin{itemize}
  \item TODO
\end{itemize}

\noindent
Wszystkie opcje linii komend \url{https://webspass.spass-prover.org/help/options.html}
\noindent \newline
Oficjana strona internetowa \url{https://webspass.spass-prover.org/}

\lstinputlisting[caption={Przykład pliku wejściowego w składni SPASS}]{listings/spass_example.in}
\lstinputlisting[caption={Przykład wyjścia SPASS}, basicstyle=\footnotesize\ttfamily]{listings/spass_example.out}

\subsection{TPTP}

TPTP - \textit{ang. (Thousands of Problems for Theorem Provers)} - to biblioteka problemów wykorzystywanych do testowania systemów \gls{ATP}. Jednocześnie jest to nazwa formatu, w którym zapisywane są te testy. TPTP udostępnia te problemy na oficjalnej stronie intenetowej. Razem z TPTP istnieje TSTP (\textit{ang. Thousands of Solutions from Theorem Provers}) - biblioteka rozwiązań problemów.
Te problemy są sklasyfikowane przez domeny (3 literowe skróty), przykładowo LCL - Logic Calculi, COL - Combinatory Logic

W formacie TPTP można zapisywać \gls{TPI}, \gls{THF}, \gls{TFF}, TCF (TODO: co to za format? Nie można go nigdzie znaleźć), \gls{FOF}, \gls{CNF}. Celem benchmarka jest badanie proverów logiku pierwszego rzędu, więc interesują nas \gls{CNF}, \gls{TFF}, TCF, \gls{FOF} (TFF/FOF with external clausifiers).


\noindent
Oficjalna strona internetowa \url{http://www.tptp.org}
\newline
Pełny spis domen \url{http://www.tptp.org/cgi-bin/SeeTPTP?Category=Documents&File=THFSynopsis}
\newline
BNF składni TPTP \url{http://www.tptp.org/TPTP/SyntaxBNF.html}

\lstinputlisting[caption={Przykład pliku w składni TPTP}]{listings/tptp_example.p}

\subsection{TPTP to LADR}

W bibliotece \gls{LADR}, która jest załączona wraz ze źródłami Provera9, dostępny jest translator składni TPTP do LADR. Jest to plik wykonywalny.

\subsection{Parser}

Zadaniem parsera jest wydobycie dodatkowych informacji statystycznych o przebiegu działana proverów, na podstawie ich wyjścia.
\newline
Każdy prover podaje inne dane na wyjściu, dostępne statystyki podane są w tabeli poniżej.
\newline
Statystyki zostaną podane w formacie json.

\begin{table}[ht]
  \centering
  \caption{Dostępne statystyki dla różnych proverów}
  \begin{tabular}{ |c|c|c| }
    \hline
    Prover & SPASS & Prover9 \\
    \hline
    SAT spełnialny & dostępny & dostępny \\
    \hline
    TODO & & \\
    \hline
  \end{tabular}
\end{table}


\subsection{Generator SAT}

Plik wykonywalny, który generuje formuły logiczne w formacie TPTP. Funkcjonuje jako osobny projekt, dlatego został opisany w sekcji \ref{LFG}

Propozycja statystyk o formule SAT: (TODO)

\begin{itemize}
  \item Number of clauses
  \item Number of atoms
  \item Maximal clause size
  \item Number of predicates
  \item Number of functors
  \item Number of variables
  \item Maximal term depth
\end{itemize}

\section{Generator formuł logicznych}\label{LFG}

Generator formuł logicznych - \textit{ang. LFG - Logic formula generator} - losowy generator formuł SAT

\subsection{Wejście}

\begin{itemize}
  \item seed - opcjonalnie - umożliwia powtarzanie losowania
  \item typ generatora liczb losowych - opcjonalne, domyślnie \textit{uniform\_int\_distribution} \url{https://en.cppreference.com/w/cpp/named_req/RandomNumberDistribution} (TODO check if needed)
  \item ilość zmiennych - wymagane, integer
  \item ilość formuł - wymagane, integer
  \item generuj \gls{SAT} w formacie \gls{CNF}
    \begin{itemize}
      \item k-SAT z zadanym k
      \item Horn-SAT (TODO what is this exaclty?)
      \item NAE3SAT (TODO what is this exaclty?)
    \end{itemize}
  \item generuj \gls{SAT} spełnialny/niespełnialny/random - TODO czy jesteśmy w stanie wygenerować formułę, która będzie wiadomo że jest spełnialna lub nie? Można użyć do badania czy formuły poprawne są rozwiązywane szybciej.
\end{itemize}

\subsection{Wyjście}

\begin{itemize}
  \item string - SAT w formacie TPTP
\end{itemize}

\subsection{Algorytm generowania \gls{CNF}}
\begin{itemize}
  \item Wypisz metadane jako komentarz (link do źródła, parametry wejściowe)
  \item TODO
\end{itemize}

\subsection{Algorytm generowania \gls{FOF}}
\begin{itemize}
  \item Wypisz metadane jako komentarz (link do źródła, parametry wejściowe)
  \item TODO
\end{itemize}

\section{Wnioski}

Przykładowe wnioski

\subsection{Ilośc zmiennych}

\subsection{Stosunek ilości zmiennych do ilości formuł}

\subsection{Formuła w postaci CNF}

\printglossary[type=\acronymtype]
\printglossary

\end{document}
