\chapter{Conclusion}

In this thesis a base for random first order logic formula generator has been presented. The generator can be easily extended for user specific needs. For example user can add arbitrary rule (by inheriting one of subgenerators) that elements must follow during generation. This enables generating problems on the fly with any complexity.
Presented solution can be extended to follow constraints related to any other \gls{FOL} element like limited number of variables, functors and so on.

The biggest challenge in described solution is randomizing formulas within given range. Solving constrains given by input parameters (as formalized in section \ref{sec:RandomizeFormulaWithinLimits} is problem of integer programming (NP problem) that is why enumerating them all is not an option. Instead random number of solutions is skipped to preserve at least pseudo-randomness. 

Alternative approach to using \gls{SMT} solver would be to take locally optimal/random decisions and introduce some randomness when taking those decisions (hit-or-miss approach). Approximate algorithm in this approach would be:

\begin{enumerate}
  \item start with empty formula
  \item add clause with random number of literals
  \item if input parameters are met - stop
  \item if input parameters are not met - go to point 2
\end{enumerate}

This algorithm has a number of edge cases which would require restarts but what is more important this approach was not used as it is hard to expand. In future development of this generator more parameter would be profided as ranges of possible value, for example user could request formula with 50 to 100 clauses, 400 to 500 literals, 70 to 80 predicates and 50 to 60 functors. A nice ability would be to suggest better parameters when user provided values are not possible.
