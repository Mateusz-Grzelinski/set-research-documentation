
\newglossaryentry{SAT} {
    name=SAT,
    first={SAT (Boolean satisfiability problem)},
    description={ \textit{ang. Boolean satisfiability problem} - problem spełnialności – zagadnienie rachunku zdań, określające czy dla danej formuły logicznej istnieje takie podstawienie (wartościowanie) zmiennych zdaniowych, żeby formuła była prawdziwa}
}

\newglossaryentry{LADR} {
    name=LADR,
    description={bilbioteka języka c, która używana jest do budowania Provera9. Określa również format składni wymagany przez Provera9}
}

\newglossaryentry{FOF} {
    name=FOF,
    first={FOF (First-order Formula)},
    description={\textit{ang. First-Order Formula} rachunek kwantyfikatorów, rachunek predykatów pierwszego rzędu. System rachunku predykatów pierwszego rzędu składa się z:
\begin{itemize}
    \item zmiennych nazwowych (litery, za które wolno podstawić nazwy dowolnych przedmiotów)
    \item stałych nazwowych (nazwy własne przedmiotów)
    \item liter predykatowych (predykaty)
    \item symboli funkcyjnych (funktory nazwotwórcze od argumentów nazwowych)
    \item stałych logicznych (spójniki prawdziwościowe rachunku zdań i kwantyfikatory)
    \item znaków pomocniczych (nawiasy)
    \item symbolu równości
\end{itemize}
    }
}

\newglossaryentry{TemporalLogic} {
    name={Logika temporalna},
    description={logika umożliwiająca rozważanie zależności czasowych bez wprowadzania czasu explicite}
}

\newglossaryentry{Prover} {
    name=Prover,
    description={inaczej SAT solver, program, który ma za zadanie rozwiązać problem SAT}
}

\newglossaryentry{CNF} {
    name=CNF,
    first={CNF (conjunctive normal form)},
    description={\textit{ang. conjunctive normal form} - koniunkcyjna postać normalna. Podzbiór FOF}
}

\newglossaryentry{ATP} {
    name=ATP,
    first={ATP (Automated Theorem Proving)},
    description={\textit{ang. Automated Theorem Proving} ATP systems are enormously powerful computer programs, capable of solving immensely difficult problems, \url{http://www.tptp.org/OverviewOfATP.html}}
}

% source http://www.tptp.org/Seminars/TPTPWorldTutorial/
\newglossaryentry{THF} {
    name=THF,
    first={THF (Typed Higher-order Logic)},
    description={\textit{ang. Typed Higher-order Logic}}
}

\newglossaryentry{TFF} {
    name=TFF,
    first={TFF (Typed First-order Logic)},
    description={\textit{ang. Typed First-order Logic}}
}

\newglossaryentry{PRP} {
    name={PRP},
    first={PRP (Propositional logic)},
    description={\textit{ang. Propositional Logic}}
}

\newglossaryentry{TPI} {
    name=TPI,
    first={TPI (TPTP Process Instruction)},
    description={\textit{ang. TPTP Process Instruction} - source \url{http://www.tptp.org/Seminars/TPI/Abstract.html}}
}

\newglossaryentry{Literal} {
    name=Literał,
    description={Jest to atom, lub jego zaprzeczenie}
}

\newglossaryentry{Atom} {
    name=Atom,
    description={formuła, która nie ma żadnych właściwych podformuł. Rodzaje formuł atomowych zależą od rodzaju używanej logiki.
    Formuły, które nie są atomowe nazywamy złożonymi. }
}

\newglossaryentry{LogicFormula} {
    name={Formuła logiczna},
    description={określenie dozwolonego wyrażenia w wielu systemach logicznych, m.in. w rachunku kwantyfikatorów oraz w rachunku zdań}
}

\newglossaryentry{ZmiennaZdaniowa} {
    name=Zmienna zdaniowa,
    description={bezargumentowy symbol w rachunku zdań. Zmiennym zdaniowym, w procesie zwanym wartościowaniem, przyporządkowywane są wartości prawda lub fałsz.}
}

\newglossaryentry{Klauzula} {
    name=Klauzula,
    description={\textit{ang. clause} – zbiór formuł logicznych. Klauzulę nazywamy prawdziwą wtedy i tylko wtedy, gdy alternatywa jej formuł logicznych jest prawdziwa. Klauzula pusta jest zawsze fałszywa. }
}

\newglossaryentry{Klauzula Horna} {
    name=Klauzula Horna,
    description={klauzula, w której co najwyżej jeden element jest niezanegowany. Przykładem takich klauzul jest ${p,\neg r,\neg q}$ }
}

\newglossaryentry{k-SAT} {
    name=k-SAT,
    description={jedna klauzula ma dokładnie k literałów, tym rozróżnia się jeszcze:
\begin{itemize}
    \item 1-SAT - jedna klauzula ma dokładnie 1 literały, np. $x \land y\land c$ - rozwiązanie w czasie liniowym
    \item 2-SAT - jedna klauzula ma dokładnie 2 literały, np. $(\neg a\lor b)\land (\neg c\lor a)\land (b\lor d)$ - rozwiązanie w P
    \item 3-SAT - jedna klauzula ma dokładnie 3 literały, np. $(\neg a\lor b\lor f)\land (\neg c\lor a\lor \neg b)\land (b\lor d\lor h)$ - rozwiązanie w NP 
    \begin{itemize}
    \item Not-all-equal 3-satisfiability (NAE3SAT) - tak jak 3-SAT, ale dodatkowo gwarantujemy, że co najmniej jeden literał w każdej klauzuli jest prawdziwy, i co najmniej jeden jest fałszywy (umożliwia to optymalizację)
    \item 1-in-3-SAT (również zwany one-in-three 3-SAT, exactly-1 3-SAT) - problemem tutaj jest tutaj, czy istnieje takie przypisanie, że dokładnie jeden literał ma wartość TRUE
    \end{itemize}
\end{itemize}
    }
}

\newglossaryentry{Term} {
    name=Term,
    description={wyrażenie składające się ze zmiennych oraz symboli
    funkcyjnych o dowolnej argumentowości (w tym o argumentowości 0, czyli
    stałych) z pewnego ustalonego zbioru. Termem może być: zmienna, atom albo
    złożenie termów}
}

\newglossaryentry{Predykat} {
    name=Predykat,
    description={ustala n-argumentową relację między jego argumentami. n może być 0. n jest nazywane argumentowością predykatu}
}

% as reference:

% \setabbreviationstyle[acronym]{long-short}
%
% \newacronym{a1}{A1}{Apples 1}
% \newacronym{a2}{A2}{Apples 2}
%
% \newglossaryentry{a}{name={A},description={Apples},alias={a1}}

% First use: \gls{a1}, \gls{a2}, \gls{a}.
% Next use: \gls{a1}, \gls{a2}, \gls{a}.
