
\newglossaryentry{SAT} {
    name=SAT,
    first={SAT (Boolean satisfiability problem)},
    description={ \textit{ang. Boolean satisfiability problem} - problem spełnialności – zagadnienie rachunku zdań, określające czy dla danej formuły logicznej istnieje takie podstawienie (wartościowanie) zmiennych zdaniowych, żeby formuła była prawdziwa}
}

\newglossaryentry{LADR} {
    name=LADR,
    description={bilbioteka języka c, która używana jest do budowania Provera9. Określa również format składni wymagany przez Provera9}
}

\newglossaryentry{FOF} {
    name=FOF,
    first={FOF (First-order Formula)},
    description={\textit{ang. First-Order Formula} reduction is a method of attempting to simplify a problem by reducing it to an equivalent set of independent subproblems. A trivial example is to reduce the conjecture A <-> B to the pair of problems A -> B and B -> A.}
}

\newglossaryentry{TemporalLogic} {
    name={Logika temporalna},
    description={logika umożliwiająca rozważanie zależności czasowych bez wprowadzania czasu explicite}
}

\newglossaryentry{Prover} {
    name=Prover,
    description={TODO}
}

\newglossaryentry{CNF} {
    name=CNF,
    first={CNF (conjunctive normal form)},
    description={\textit{ang. conjunctive normal form} - koniunkcyjna postać normalna}
}

\newglossaryentry{ATP} {
    name=ATP,
    first={ATP (Automated Theorem Proving)},
    description={\textit{ang. Automated Theorem Proving} ATP systems are enormously powerful computer programs, capable of solving immensely difficult problems, \url{http://www.tptp.org/OverviewOfATP.html}}
}

% source http://www.tptp.org/Seminars/TPTPWorldTutorial/
\newglossaryentry{THF} {
    name=THF,
    first={THF (Typed Higher-order Logic)},
    description={\textit{ang. Typed Higher-order Logic}}
}

\newglossaryentry{TFF} {
    name=TFF,
    first={TFF (Typed First-order Logic)},
    description={\textit{ang. Typed First-order Logic}}
}

\newglossaryentry{PRP} {
    name={PRP},
    first={PRP (Propositional logic)},
    description={\textit{ang. Propositional Logic}}
}

\newglossaryentry{TPI} {
    name=TPI,
    first={TPI (TPTP Process Instruction)},
    description={\textit{ang. TPTP Process Instruction} - source \url{http://www.tptp.org/Seminars/TPI/Abstract.html}}
}

% as reference:

% \setabbreviationstyle[acronym]{long-short}
%
% \newacronym{a1}{A1}{Apples 1}
% \newacronym{a2}{A2}{Apples 2}
%
% \newglossaryentry{a}{name={A},description={Apples},alias={a1}}

% First use: \gls{a1}, \gls{a2}, \gls{a}.
% Next use: \gls{a1}, \gls{a2}, \gls{a}.
