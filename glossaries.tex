
\newglossaryentry{SAT} {
    name=SAT,
    first={SAT (Boolean Satisfiability Problem)},
    description={Boolean Satisfiability Problem - problem spełnialności – zagadnienie rachunku zdań, określające czy dla danej formuły logicznej istnieje takie podstawienie (wartościowanie) zmiennych zdaniowych, żeby formuła była prawdziwa}
}

\newglossaryentry{LADR} {
    name=LADR,
    description={bilbioteka języka c, która używana jest do budowania Provera9. Określa również format składni wymagany przez Provera9}
}

\newglossaryentry{BNF} {
    name=BNF,
    first={BNF (Backus normal form)},
    description={Backus normal form}
}

\newglossaryentry{FOF} {
    name=FOF,
    first={FOF (First Order Formula)},
    description={First Order Formula}
}

\newglossaryentry{FOL} {
    name=FOL,
    first={FOL (First Order Logic)},
    description={First Order Logic}
}

\newglossaryentry{TL} {
    name={Temporal Logic},
    first={TL (Temporal Logic)},
    description={Temporal Logic}
}

\newglossaryentry{prover} {
    name=prover,
    description={inaczej SAT solver, program, który ma za zadanie rozwiązać problem SAT}
}

\newglossaryentry{CNF} {
    name=CNF,
    first={CNF (Conjunctive Normal Form)},
    description={Conjunctive Normal Form}
}

\newglossaryentry{ATP} {
    name=ATP,
    first={ATP (Automated Theorem Proving)},
    description={Automated Theorem Proving. ATP systems are enormously powerful computer programs, capable of solving immensely difficult problems, \url{http://www.tptp.org/OverviewOfATP.html}}
}

% source http://www.tptp.org/Seminars/TPTPWorldTutorial/
\newglossaryentry{THF} {
    name=THF,
    first={THF (Typed Higher-order Logic)},
    description={Typed Higher-order Logic}
}

\newglossaryentry{TFF} {
    name=TFF,
    first={TFF (Typed First-order Logic)},
    description={Typed First-order Logic}
}

\newglossaryentry{PRP} {
    name={PRP},
    first={PRP (Propositional logic)},
    description={Propositional Logic}
}

\newglossaryentry{TPI} {
    name=TPI,
    first={TPI (TPTP Process Instruction)},
    description={TPTP Process Instruction - source \url{http://www.tptp.org/Seminars/TPI/Abstract.html}}
}

\newglossaryentry{LTL} {
    name=LTL,
    first={LTL (Linear temporal logic)},
    description={jedna z logit temporalnych}
}

\newglossaryentry{Literal} {
    name=Literał,
    description={Jest to atom, lub jego zaprzeczenie}
}

\newglossaryentry{Atom} {
    name=Atom,
    description={formuła, która nie ma żadnych właściwych podformuł. Rodzaje formuł atomowych zależą od rodzaju używanej logiki.
    Formuły, które nie są atomowe nazywamy złożonymi. }
}

\newglossaryentry{LogicFormula} {
    name={Formuła logiczna},
    description={określenie dozwolonego wyrażenia w wielu systemach logicznych, m.in. w rachunku kwantyfikatorów oraz w rachunku zdań}
}

\newglossaryentry{ZmiennaZdaniowa} {
    name=Zmienna zdaniowa,
    description={bezargumentowy symbol w rachunku zdań. Zmiennym zdaniowym, w procesie zwanym wartościowaniem, przyporządkowywane są wartości prawda lub fałsz.}
}

\newglossaryentry{Klauzula} {
    name=Klauzula,
    description={clause – zbiór formuł logicznych. Klauzulę nazywamy prawdziwą wtedy i tylko wtedy, gdy alternatywa jej formuł logicznych jest prawdziwa. Klauzula pusta jest zawsze fałszywa. }
}

\newglossaryentry{Klauzula Horna} {
    name=Klauzula Horna,
    description={klauzula, w której co najwyżej jeden element jest niezanegowany. Przykładem takich klauzul jest ${p,\neg r,\neg q}$ }
}

\newglossaryentry{k-SAT} {
    name=k-SAT,
    description={jedna klauzula ma dokładnie k literałów, tym rozróżnia się jeszcze:
\begin{itemize}
    \item 1-SAT - jedna klauzula ma dokładnie 1 literały, np. $x \land y\land c$ - rozwiązanie w czasie liniowym
    \item 2-SAT - jedna klauzula ma dokładnie 2 literały, np. $(\neg a\lor b)\land (\neg c\lor a)\land (b\lor d)$ - rozwiązanie w P
    \item 3-SAT - jedna klauzula ma dokładnie 3 literały, np. $(\neg a\lor b\lor f)\land (\neg c\lor a\lor \neg b)\land (b\lor d\lor h)$ - rozwiązanie w NP 
    \begin{itemize}
    \item Not-all-equal 3-satisfiability (NAE3SAT) - tak jak 3-SAT, ale dodatkowo gwarantujemy, że co najmniej jeden literał w każdej klauzuli jest prawdziwy, i co najmniej jeden jest fałszywy (umożliwia to optymalizację)
    \item 1-in-3-SAT (również zwany one-in-three 3-SAT, exactly-1 3-SAT) - problemem tutaj jest tutaj, czy istnieje takie przypisanie, że dokładnie jeden literał ma wartość TRUE
    \end{itemize}
\end{itemize}
    }
}

\newglossaryentry{Term} {
    name=Term,
    description={wyrażenie składające się ze zmiennych oraz symboli funkcyjnych o dowolnej argumentowości (w tym o argumentowości 0, czyli stałych) z pewnego ustalonego zbioru. Termem może być: zmienna, atom albo złożenie termów}
}

\newglossaryentry{Predykat} {
    name=Predykat,
    description={ustala n-argumentową relację między jego argumentami. n może być 0. n jest nazywane argumentowością predykatu}
}

\newglossaryentry{TPTP2X} {
    name=TPTP2X,
    first={TPTP2X},
    description={The TPTP2X utility is a multi-functional utility for reformatting, transforming, and generating TPTP problem files.}
}

\newglossaryentry{TPTP4X} {
    name=TPTP4X,
    first={TPTP4X (Thousands of Problems for Theorem Provers)},
    description={The TPTP4X utility is a multi-functional utility for reformatting, transforming, and generating TPTP problem files. It is the successor to the TPTP2X utility, and offers some of the same functionality, and some extra functionality. TPTP4X is written in C, and is thus faster than TPTP2X.}
}

\newglossaryentry{TPTP} {
    name=TPTP,
    first={TPTP (Thousands of Problems for Theorem Provers)},
    description={Thousands of Problems for Theorem Provers, official wesite \url{http://www.tptp.org}}
}

\newglossaryentry{TSTP} {
    name=TSTP,
    first={TSTP (Thousands of Solutions from Theorem Provers)},
    description={Thousands of Solutions from Theorem Provers}
}

\newglossaryentry{SMTLIB} {
    name=SMTLIB,
    first={SMTLIB (Satisfiability Modulo theories (SMT) library)},
  description={SMT-LIB is an international initiative aimed at facilitating research and development in Satisfiability Modulo Theories (SMT). Homepage \url{http://smtlib.cs.uiowa.edu/index.shtml}}
}

% as reference:

% \setabbreviationstyle[acronym]{long-short}
%
% \newacronym{a1}{A1}{Apples 1}
% \newacronym{a2}{A2}{Apples 2}
%
% \newglossaryentry{a}{name={A},description={Apples},alias={a1}}

% First use: \gls{a1}, \gls{a2}, \gls{a}.
% Next use: \gls{a1}, \gls{a2}, \gls{a}.
