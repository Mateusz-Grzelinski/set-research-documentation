\documentclass[a4paper,12pt]{article}
\usepackage{amsmath}
%\usepackage{polish}
\usepackage[polish]{babel}
\usepackage[utf8]{inputenc}
\usepackage[T1]{fontenc}
\usepackage{graphicx}
\usepackage{anysize}
\usepackage{enumerate}
\usepackage{times}
\usepackage{plain}
\usepackage{caption}
\usepackage{graphicx}
\usepackage[space]{grffile}
\usepackage{setspace}
\usepackage{multirow}
\usepackage{datetime}
\usepackage[outputdir=build]{minted}
\usepackage{xcolor}
\usepackage[colorlinks=true]{hyperref}
\usepackage[automake, acronyms, toc, nopostdot, nonumberlist, nomain]{glossaries}
\usepackage{indentfirst}
\hypersetup{
  colorlinks=true,
  linkcolor=blue,
  filecolor=magenta,
  urlcolor=cyan,
}
\urlstyle{same}
%\marginsize{left}{right}{top}{bottom}
\marginsize{2.5cm}{2.5cm}{2.5cm}{2.5cm}
\setlength{\parindent}{4em}
\setlength{\parskip}{1em}
\renewcommand{\baselinestretch}{2.0}

% listings with page breaking
\newenvironment{longlisting}{\captionsetup{type=listing}}{}

\definecolor{bg}{rgb}{0.95,0.95,0.95}
\setminted{
  fontfamily=txtt,
  fontsize=\footnotesize,
  samepage=false,
  style=xcode,
  breaklines,
  bgcolor=bg
}

% custom lexer for toml
\newminted[tomlcode]{../lexers/toml.py:TomlLexer -x}{}
\newmintinline[tomlcodeinline]{../lexers/toml.py:TomlLexer -x}{}
\newmintedfile[tomlfile]{../lexers/toml.py:TomlLexer -x}{}

% custom lexer for ladr
\newminted[ladrcode]{../lexers/ladr.py:LadrLexer -x}{}
\newmintinline[ladrcodeinline]{../lexers/ladr.py:LadrLexer -x}{}
\newmintedfile[ladrfile]{../lexers/ladr.py:LadrLexer -x}{}

% custom lexer for spass
\newminted[spasscode]{../lexers/spass.py:SpassLexer -x}{}
\newmintinline[spasscodeinline]{../lexers/Spass.py:SpassLexer -x}{}
\newmintedfile[spassfile]{../lexers/spass.py:SpassLexer -x}{}

% custom lexer for tptpt
\newminted[tptpcode]{../lexers/tptp.py:TptpLexer -x}{}
\newmintinline[tptpcodeinline]{../lexers/tptp.py:TptpLexer -x}{}
\newmintedfile[tptpfile]{../lexers/tptp.py:TptpLexer -x}{}

\loadglsentries{glossaries}
\makeglossaries
\glsaddall

\begin{document}
\onehalfspacing
\begin{figure}[!htb]
  \centerline{\includegraphics[scale=0.8]{../images/agh_logo.jpg}}
\end{figure}

\begin{center}
  \Huge{Studio projektowe 2\\}
  \Large{Benchmark solvera InKreSAT\\ \large \textit \today \\}
  \vspace{3cm}
  \Large{	Autorzy:\\
    Mateusz Grzeliński\\
    Przemysław Michałek\\
  }
  \large{Wydział Elektrotechniki, Automatyki, Informatyki i Elektroniki}

  \newpage
\end{center}

\tableofcontents
\newpage

\section{Wprowadzenie}

Celem projektu jest zbadanie wydajności automatyczych metod dowodzenia twierdzeń InKreSAT w kontekście formuł żywotnościowych i bezpieczeństwa, przedstawionych jako problem \gls{SAT}, a dokładniej \gls{TL}. Na początku zostaje wygenerowna formuła \gls{SAT}, która zostaje rozwiązana przez badany prover. Badany jest czas wykonania, rezultat (czy \gls{SAT} jest spełnialny), użycie pamięci RAM.  Generowana formuła \gls{SAT} jest modyfikowana ze względu na między innymi długość formuły, ilość zmiennych.

Prover traktowany jest jako czarna skrzynka (blackbox), jego parametry są modyfikowane z poziomu linii komend[???].

\subsection{Żywotność i bezpieczeństwo formuł}

\textbf{Żywotność} systemu to cecha, która zapewnia, że coś dobrego na pewno w końcu się wydarzy. Formuła żywotnościowa gwarantuje, że istnieje co najmniej jedno wydarzenie, dla którego formuła będzie spełniona.

\textbf{Bezpieczeństwo} systemu to cecha, która zapewnia, że nic złego nigdy się nie stanie. Formuła bezpieczeństwa gwarantuje, że dla każdego wydarzenia, formuła bezpieczeństwa nigdy nie zostanie pogwałcona.

Żywotność oraz bezpieczeństwo mogą zostać wyrażone na przykład w postaci formuły logicznej pierwszego rzędu, czyli problemu SAT.
W formacie TPTP formuły żywotnościowe i bezpieczeństwa można przedstawić jako:
\begin{itemize}
  \item kwantyfikatory (\gls{FOF}) - w tptp kwantyfikator uniwersalny to \mintinline{text}{!}, a kwantyfikator egzystencjalny \mintinline{text}{?}
  \item kaluzule \gls{CNF} - powstaje po przekonwertowaniu formuły z kwantyfikatorem w klazulę
\end{itemize}

\begin{tptpcode}
fof(simple_exists, axiom,
 ? [W,Z] :  p(W, Z) | p(a, b)
  ).

% converted with TPTP2X, otter algorithm
cnf(simple_exists_1,axiom,
    ( p(sk1,sk2) | p(a,b) )).
\end{tptpcode}

\begin{tptpcode}
fof(simple_for_all, axiom,
 ! [W,Z] :  p(W, Z) | p(a, b)
  ).

% converted with TPTP2X, otter algorithm
cnf(simple_for_all_1,axiom,
    ( p(A,B) | p(a,b) )).
\end{tptpcode}

\begin{tptpcode}
% dla każdego X, Y operacja lesseq, to to samo co less lub równość
fof(this_is_obvious, axiom,
  ! [X,Y] : ( $lesseq(X,Y) <=> ( $less(X,Y) | X = Y ) )
  ).

% converted with TPTP2X, otter algorithm
cnf(this_is_obvious_1,axiom,
    ( ~ $lesseq(A,B) | $less(A,B) | A = B )).

cnf(this_is_obvious_2,axiom,
    ( ~ $less(A,B) | $lesseq(A,B) )).

cnf(this_is_obvious_3,axiom,
    ( A != B | $lesseq(A,B) )).
\end{tptpcode}

\begin{tptpcode}
fof(combined, axiom,
 ? [W,Z] : ( ! [X, Y] : p(W, Z, X)  | d(Y) )
  ).

% converted with TPTP2X, otter algorithm
cnf(combined_1,axiom,
    ( p(sk1,sk2,A) | d(B) )).
\end{tptpcode}

\subsubsection{Przykład: zdawanie egzaminu}

\noindent
Problem: zalicz egzamin, aby zdać kurs

\noindent
Warunek bezpieczeństwa: jeżeli podejmujesz się egzaminu, zalicz go

\noindent
Warunek żywotnościowy: kiedyś musisz podejść do egzaminu

\subsubsection{Przykład: światła na skrzyżowaniu}

\noindent
Problem: samochody chcą przejechać przez skrzyżowanie

\noindent
Warunek bezpieczeństwa: tylko jedno światło powinno być zielone

\noindent
Warunek żywotnościowy: każde światło powinno kiedyć zmienić się na zielone


\section{Benchmark}

Problem benchmarku w strategii blackbox sprowadza sie do wykonania podprogramu z odpowiednimi opcjami z poziomu linii komend.
Wejście oraz testy benchmarka ustawiane są poprzez plik konfiguracyjny, sekcja \ref{benchmarkUsage}.  Wejściem testu jest zbiór formuł \gls{SAT} zapisanych na dysku w formacie TPTP w postaci pliku tesktowego. Test polega na podaniu pliku TPTP do provera. W razie potrzeby nastąpi automatyczna konwersja do odpowiedniej składni za pomocą dostępnych translatorów.
Dla każdego provera dostępne są statystyki:

\begin{itemize}
  \item czas wykonania
  \item użycie pamięci RAM
  \item spełnialność formuły
\end{itemize}

\noindent
Więcej statystyk może zostać zebrany przez parser, który bada wyjście provera (sekcja \ref{parser}).
\newline
Zbadane zostaną statystyki proverów ze względu na następujące właściwości formuły SAT:

% more here:http://www.tptp.org/TPTP/TR/TPTPTR.shtml#ProblemGenerators
\begin{itemize}
  \item SAT type (CNF, FOF)
  \item number of clauses (CNF) - w składni TPTP jest to liczba użytych słów \mintinline{text}{cnf}
  \item number of atoms - ilość literałów połączonych operatorem \textit{lub}
  \item maximal clause size
  \item number of predicates - predykat ustanawia n-argumentową relację między argumentami
  \item number of functors
  \item number of variables - ilość atomów, które zaczynają się dużą literą. Jeśli zmienna \textit{X} wystąpi w dwóch różnych klauzulach, traktowana jest jako 2 różne zmienne
  \item maximal term depth
\end{itemize}

Example in tptp syntax:

\begin{tptpcode}
% total in this example: 2 clauses, 6 atoms, 4 predicates, 1 functor, 2 variables

% clause: 4 atoms, 4 predicates, 0 functors, 1 variable
cnf(predicates_examples, axiom,
  ( 'p 1' % arity 0
  | p2
  | 'p 3'(X) % arity 1, variable X
  | p4(y) % arity 1, fact y
  )).

% clause: 2 atoms, 1 predicate, 1 functor, 1 variable
cnf(clause2, negated_conjecture,
    ( p4(X)
    | p4(f(X)) % f(X) is a functor
    )).

\end{tptpcode}

%\begin{figure}[H]
%  \centering
%  \includegraphics[scale=0.5]{benchmark/components.png}
%  \caption{Diagram komponentów systemu benchmarka}
%\end{figure}

\subsection{InKreSAT}

InKreSAT to zautomatyzowane narzędzie udowadniające dla logiki temporalnej stworzone przez Marka Kaminskiego oraz Tobiasa Tebbi.

InKreSAT dostępny jest jako plik wykonywalny, przyjmuje pliki w formacie \gls{LADR}. Dla uniwersalności zostanie zastosowany konwerter TPTP do LADR dostępny wraz z InKreSAT jako osobny plik wykonywalny. Ilość opcji dostępnych z lini komend jest minimanlna, jedyną ważną opcją z punktu widzenia benchmarka jest \mintinline{text}{-x - set(auto2).  (enhanced auto mode)}

\noindent
Oficjalna strona internetowa \url{https://www.ps.uni-saarland.de/~kaminski/inkresat/}


[TO DO]
%\begin{longlisting}
%  \caption{Przykład pliku wejściowego w składni LADR}
%  \ladrfile{listings/prover9_example.in}
%\end{longlisting}
%
%\begin{longlisting}
%  \caption{Przykład wyjścia Provera9}
%  \inputminted{text}{listings/prover9_example.out}
%\end{longlisting}
[/TO DO]

\subsection{Parser} \label{parser}

Zadaniem parsera jest wydobycie dodatkowych informacji statystycznych o przebiegu działana proverów, na podstawie ich wyjścia.
\newline
Każdy prover podaje inne dane na wyjściu, dostępne statystyki podane są w tabeli poniżej.
\newline
Statystyki zostaną podane w formacie json.

\begin{table}[ht]
  \centering
  \caption{Dostępne statystyki dla różnych proverów}
  \begin{tabular}{ |c|c|c| }
    \hline
    Prover & SPASS & Prover9 \\
    \hline
    SAT spełnialny & dostępny & dostępny \\
    \hline
    TODO & & \\
    \hline
  \end{tabular}
\end{table}

\subsection{Użycie i konfiguracja} \label{benchmarkUsage}

Ze względu na mnogośc opcji, większość opcji zawarta jest w pliku konfiguracyjnym \ref{configFile} w formacie \textit{toml}.

Najpierw definiowana jest lista wejść (\tomlcodeinline{testInput}). Wejście to zbiór plików, które można jednoznacznie zidentyfikować za pomocą nazwy (\tomlcodeinline{name}).
Następnie definiowana jest lista zestawów testowych (\tomlcodeinline{testSuite}). Zestaw testowy definiuje parametry wspólne dla kilku przypadków testowych (\tomlcodeinline{testCase}) np. ścieżka do pliku wykonywalnego. Każdy zestaw testowy posiada listę przypadków testowych. Każdy przypadek testowy definiuje w jakim formacie oczekuje wejście. Jeśli formaty są różne, konflikt jest rozwiązywany przy pomocy dostępnych translatorów. Opcje do testowania są definiowane jako lista. Plik wejściowy może zosać podany przez standardowe wejście, przez opcję lini komend lub jako ostatni argument w komendzie.

\begin{minted}{bash}
testSuite.executable testSuite.options testSuite.testCase.options [input_after_option file_path] [file_path]
\end{minted}

\subsubsection{Wspierane funkconalności}
[TO VERIFY]
\begin{itemize}
  \item ścieżka do pliku wykonywalnego może być podana w pliku konfiguracyjnym, lub może być zawarta w zmiennyj środowiskowej \mintinline{text}{PATH} (ścieżka w pliku ma pierwszeństwo)
  \item definiowanie opcji linii komend do testowania pliku wykonywalnego
  \item definiowanie listy źródeł do testów. Źródłem do testów mogą być tylko pliki tesktowe
  \item definiowanie które wejścia mają byś przetestowane w przypadku testowym
    \begin{itemize}
      \item testuj tylko wymienione - opcja \tomlcodeinline{include_only},
      \item testuj wszystkich oprócz - opcja \tomlcodeinline{exclude},
      \item testuj wszystkie zdefiniowane wejścia - nie podając żadnej z opcji
    \end{itemize}
  \item pozycja nazwy pliku źródłowego może być ustawiona w następujący sposób
    \begin{itemize}
      \item domyślnie plik podawany jest na standardowe wejście
      \item podaj plik jako ostatni argument \tomlcodeinline{input_as_last_argument}
      \item podaj plik jako argument po opcji \tomlcodeinline{input_after_option}
    \end{itemize}
  \item TODO: wyniki zapisywane są jako plik \textit{json} do katalogu wyściowego zdefiniowanego w pliku konfiguracyjnym.
\end{itemize}

\subsubsection{Ograniczenia}

\begin{itemize}
  \item podanie kilku plików wejściowych naraz dla jednej testownej komendy nie jest możliwe, np. \mintinline{text}{-o file1.p -o file2.p}
\end{itemize}

\begin{longlisting}
  \caption{Przykład pliku konfiguracyjnego benchmarka}
  \label{configFile}
  \tomlfile{benchmark/example_config.toml}
\end{longlisting}


\begin{longlisting}
  \caption{Przykładowe komendy testowe}
  \begin{tomlcode}
# ...
[[testSuites]]
executable="ls"
options=["-1"]
# ...

[[testSuites.testCases]]
options=[""]
# test cases:
# ls -1

[[testSuites.testCases]]
options=["", "-r -l"]
# test cases:
# ls -1
# ls -1 -r -l
  \end{tomlcode}
\end{longlisting}
[/TO VERIFY]

\subsection{Diagramy}
%
%\begin{figure}[H]
%  \centering
%  \includegraphics[width=0.9\textwidth]{benchmark/class_diagram.png}
%  \caption{Diagram klas}
%\end{figure}
%
%\begin{figure}[H]
%  \centering
%  \includegraphics[width=\textwidth]{benchmark/activity_diagrams/main_run.png}
%  \caption{Diagram aktywności}
%\end{figure}
%
%\begin{figure}[H]
%  \centering
%  \includegraphics[width=0.8\textwidth]{benchmark/activity_diagrams/benchmark_run.png}
%  \caption{Diagram aktywnośći}
%\end{figure}
%
%\begin{figure}[H]
%  \centering
%  \includegraphics[width=\textwidth]{benchmark/activity_diagrams/test_suite_run.png}
%  \caption{Diagram aktywnośći}
%\end{figure}
%
%\begin{figure}[H]
%  \centering
%  \includegraphics[width=\textwidth]{benchmark/activity_diagrams/test_case_run.png}
%  \caption{Diagram aktywnośći}
%\end{figure}

\section{Generator formuł logicznych} \label{LFG}

Losowy generator formuł SAT. Generator może generować formuły \gls{CNF} oraz \gls{k-SAT}. Generator może generować formuły w formacie DIMACS oraz TPTP.

%\begin{figure}[H]
%  \centering
%  \includegraphics[scale=0.7]{logic-formula-generator/components.png}
%  \caption{Komponenty generatora CNF}
%\end{figure}
%
%\begin{figure}[H]
%  \centering
%  \includegraphics[scale=0.7]{logic-formula-generator/cnf_class_diagram.png}
%  \caption{Diagram klas generatora CNF}
%\end{figure}

\noindent
Założenia ogólne:
\begin{itemize}
  \item na jedną formułę, składa się $w$ niezależnych grup kauzul ($w$ z przedziału $[1, \infty]$). Wygenerowanie jednej grupy klauzul polega na ponownym uruchomieniu generatora z innymi parametrami. Przy czym:
    \begin{itemize}
      \item grupa klauzul nie współdzieli zbioru zmiennych
      \item grupa może posiadać całkowicie inne parametry sterujące
    \end{itemize}
  \item po wygenerowaniu, grupa klazul może być wymieszana z inną grupą klauzul. Mieszanie polega na:
    \begin{itemize}
      \item utworzeniu nowych klauzul bazując na zmiennych występujących w obu grupach - wprowadza to nowe zależności między istniejącymi już zmiennymi
    \end{itemize}
\end{itemize}

\noindent
Szczegóły implementacyjne:
\begin{itemize}
  \item logika generatora (backend) używa ogranieczeń sztywnych. Na przykład, generator zmiennych musi wygenerować \textbf{dokładnie} $n$ zmiennych, generator klauzul musi wygenerować \textbf{dokładnie} $m$ klauzul itp.
  \item ograniecznia miękkie, tzn. wygeneruj formułę z \textbf{około} $n$ zmiennych i \textbf{około} $m$ klauzul, uzyskiwane są przez wygenerowanie dokładnych warości na frontendzie i uruchomienie generatora z dokładnymi wartościami
  \item generatory są obiektami tylko do odczytu
\end{itemize}

\subsection{Generator zmiennych (\textit{VariableGenerator})}

Generator zmiennych ma za zadanie podanie dokładnie $n$ zmiennych, w tym $m$ różnych.

\noindent
Parametry sterujące:

\begin{itemize}
  \item nazwa zmiennej
  \item $m$ - ilość różnych zmiennych
  \item $n$ - ilość zmiennych do wygenerowania, ilość różnych zmiennych jest mniejsza lub równa ilości zmiennych
  \item prawdopodobieństwo zanegowania zmiennej
\end{itemize}

\noindent
Ograniczenia:
\begin{enumerate}
  \item prawdopodobieństwo zanegowania jest z zakresu $[0,1]$
  \item ilość zmiennych jest większa lub równa niż ilość różnych zmiennych $n>=m$
\end{enumerate}

\subsection{Generator klauzul (\textit{ClauseGenerator})} \label{ClauseGenerator}

Bazując na zmiennych podanych przez generator zmiennych, generuje dokładnie $k$ klauzul. Generator zmiennych musi zostać całkowicie wyczerpany. Klauzule które pochodzą z jednego generatora, nazywane są grupą klauzul.

\noindent
Parametry sterujące:
\begin{itemize}
  \item generator zmiennych
  \item ilość klauzuj do wygenerowania
  \item maksymalny rozmiar klauzuli
\end{itemize}

\noindent
Ograniczenia:
\begin{enumerate}
  \item na klauzulę przypada co najmniej jedna zmienna
  \item ilość zmiennych wymaganych przez generator klauzul musi być większa niż ilość zmiennych dostaczanych przez generator zmiennych (cały generator zmiennych musi zostać skonsumowany)
  \item każdy literał w pojedyńczej kaluzuli jest różny - nie może wystąpić: $(p1 \lor p1) \land (p1 \lor p2)$
  \item każda klauzula jest różna - nie może wystąpić: $(p1 \lor p2) \land (p1 \lor p2)$
\end{enumerate}

\subsection{Formuła (\textit{Formula})}

Formuła powstaje w wyniku kilkukrotnego uruchomiania generatora klauzul \ref{ClauseGenerator}. Zbiera ona wszystkie grupy klauzul i umożliwia mieszanie ich.

[TO DO]
\section{Logika pierwszego rzędu - nazewnictwo w kontekście TPTP}

\textbf{Term}
to zmienna, stała lub wynik działania funktorów na zmiennych i stałych

\textbf{Atom}
to wyrażenie logiczne, które nie może zostać rozbite na składowe

% \textbf{Equolity atom}

\textbf{Literał}
to atom lub jego zaprzeczenie

\textbf{Zmienna}
to atom, który zaczyna się z dużej litery. Zmienna ma zasięg klauzulu. Tzn. jeśli zmienna $A$ pojawi się w jednej klazuli kilkukrotnie jest to dalej jedna zmienna.

\textbf{Zmienna singletonowa}
to zmienna użyta tylko raz w klauzuli

\textbf{Unit clause}
to klauzula, która posiada tylko jeden atom

\textbf{Horn clause}
to klaulula, która posiada co najmniej jeden pozytywny literał

\textbf{RR clause} - ??

\textbf{Predykat}
jest to operator logiczny, który zwraca prawdę lub fałsz. Predykat operuje na określonej liczbie termów. Liczba ta jest stała i nazywana argumentowością predykatu (\textit{arity}).

\textbf{Funktor}
oprator logiczny, który zwraca term. Funktor posiada określoną argumentowość.

\textbf{Constant functor}
to funktor o argumentowości 0

\textbf{Klauzula}
jest to zbiór termów połączonych dysjunkcją ($\lor$)

% \textbf{Kwantyfikator} not in scope

\begin{tptpcode}
% TPTP CNF formula example 1
cnf(simple_clause_1, axiom,
    ( p(f,f) | ~p(a,b) | p(X, V) | pp(X) )).
 
% powyższa formuła składa się z:
% 1 klauzuli: w tym 0 jednostkowa, 1 Horn
% 4 literały: [p(f,f), ~p(a,b), p(X, V), pp(X)]
% 4 atomy: [p(f,f), p(a,b), p(X, V), pp(X)]
% 2 predykatów: [p, pp] o argumentowości 1: [pp] i 2: [p]
% 3 funktorów: [f, a, b] o argumentowości 0 - funktory stałe
% 2 zmiennych: [X, V] w tym 1 zmienna singletonowo
\end{tptpcode}

\begin{tptpcode}
% TPTP CNF formula example 2
cnf(simple_clause_1, axiom,
    ( p(f,f) | ~p(a,b) | p(X, V) | pp(X) )).

cnf(simple_clause_2, axiom,
    ( pp(f) | pp(X) )).

cnf(simple_clause_3, axiom,
    ( ppp )).

% powyższa formuła składa się z:
% 3 klauzuli, w tym 1 klauzula jednostkowa, 3 Horn
% 4 literały: [p(f,f), ~p(a,b), p(X, V), pp(X)]
% 4 atomy: [p(f,f), p(a,b), p(X, V), pp(X)]
% 3 predykatów: [p, pp, ppp] o argumentowości 0: [ppp], 1: [pp] i 2: [p]
% 3 funktorów: [f, a, b] o argumentowości 0 - funktory stałe
% 3 zmiennych: [X, V] w tym 2 zmienne singletonowe
\end{tptpcode}

\begin{tptpcode}
% TPTP CNF formula example 3
cnf(simple_clause_1, axiom,
    ( p(f,f) | ~p(f,f) )).

% powyższa formuła składa się z:
% 1 klauzuli, w tym 0 jednowtkowa, 1 Horn
% 2 literały: [p(f,f), ~p(f,f)]
% 1 atomy: [p(f,f)]
% 1 predykatu: [p] o argumentowości 2
% 1 funktorów: [f] o argumentowości 0 - funktor stały
% 0 zmiennych
\end{tptpcode}

\begin{tptpcode}
% TPTP CNF formula example 4
cnf(simple_clause_1, axiom,
    ( p(f,f) | ~p(f,f) )).

cnf(simple_clause_2, axiom,
    ( ~p(f,f) )).

% powyższa formuła składa się z:
% 2 klauzuli, w tym 1 jednostkowa, 1 Horn
% 2 literały: [p(f,f), ~p(f,f)]
% 1 atomy: [p(f,f)]
% 1 predykatu: [p] o argumentowości 2
% 1 funktorów: [f] o argumentowości 0 - funktor stały
% 0 zmiennych
\end{tptpcode}
[/TO DO]
\newpage
\section{Zestaw testowy}


\noindent
\textbf{Problem:} Jak ilość atomów wpływa na czas wykonania?

Zestaw to formuły CNF różnej długości
\begin{itemize}
  \item liczba klauzul to 100, 200, 500, 1000, 2500, 5000
  \item stosunek atomów do klauzul wynosi 2, 3, 4, 5, 10 (2 oznacza, że jeśli liczba klauzul to 100, wtedy liczba atomów to 200)
  \item maksymalna długość klauzuli to ilość wszystkich zmiennych / ilość klauzul
\end{itemize}

\noindent
\textbf{Problem:} Jak stosunek atomów bezpieczeństwa do atomów żywotnościowych wpływa na czas wykonania?

Zestaw to formuły CNF różnej długości
\begin{itemize}
  \item liczba klauzul to 100, 200, 500, 1000, 2500, 5000
  \item stosunek atomów bezpieczeństwa do atomów żywotnościowych wynosi 2, 3, 4, 5, 10
  \item maksymalna długość klauzuli to ilość wszystkich zmiennych / ilość klauzul
\end{itemize}

\noindent
\textbf{Problem:} Jak udział k-SAT i zmiennych wpływa na czas wykonania?
\newline
\textbf{Hipoteza:}
\begin{enumerate}
  \item obecność co najmniej jednej klauzul k-SAT, gdzie k dąży do 1, sprzyja szybkiemu rozwiązaniu problemu
  \item obecność co najmniej jednej klauzul k-SAT, gdzie k dąży do $\infty$, znacząco spowalnia rozwiązanie problemu
  \item im większa obecność klazul z dużym k, tym większy czas wykonywania
\end{enumerate}

Zestaw skupia się na k-SAT:
\begin{itemize}
  \item liczba klauzul to 100, 200, 300, 400, 500, 1000, 2000, 3000, 4000, 5000
  \item klauzule są postaci:
    \begin{itemize}
      \item 1,5,10,20-SAT -- kada z nich stanowi 25\% wszystkich klauzul (po równo)
      \item 1,5,10,20-SAT -- 1-SAT to 1\% (minimum 1), 5,10,20-SAT po równo
      \item 1,5,10-SAT -- po równo
      \item 1,5,10,20-SAT -- 20-SAT to 1\% (minimum 1), 1,5,10-SAT po równo
      \item 5,10,20-SAT -- po równo
    \end{itemize}
\end{itemize}

% \noindent
% \textbf{•}f{Problem:} Jak zmiesznie klauzul i zmiennych wpływa na czas wykonania? TODO nie wiemy co badać

% Zestaw stworzony jest z mieszania niezależnych grup klauzul (grupy nie współdzielą zmiennych).  Mieszanie polega na dodaniu klauzul, które operują na wspólnym zbiorze zmiennych z obu grup.
% \begin{itemize}
%   \item grupa posiada 50 zmiennych
%   \item liczba grup to 5, 10, 15, 20
%   \item mieszanie:
%     \begin{itemize}
%       \item bez mieszania (grupa kontrolna)
%       \item weź grupy o numerach (1, 2) i stwórz jeszcze 10 klazul, ilość zmiennych jest losowa. Powtórz to samo dla grup $(n,n+1)$ - jedna grupa jest powiązana tylko z jedną grupą
%       \item inne kombinacje, co chcemy z tego wywnioskować?
%     \end{itemize}
%   \item w wyniku zmieszania powstaje od 2 do 25 nowych klauzul. Do stworzenia klauzul użytych jest od 20 do 50 literałów
% \end{itemize}
%
\section{Wnioski}

Celem projektu było zbadanie które z czynników formuł logicznych wpływają najbardziej na działanie provera InKreSAT. Aby odpowiedzieć na to pytanie napisany został generator formuł o zadanych parametrach które następnie były przepuszczane przez prover z użyciem strategii blackbox. Na wyjściu otrzymane zostały pliki JSON z parametrami wejścia takimi jak: liczba klauzul, atomów, predykatów, funktorów, zmiennych oraz parametrami wyjścia: czasem wykonania, maksymalnym zużyciem pamięci (peak memory) oraz statusem (spełnialna, niespełnialna, timeout). W celu ograniczenia czasu wykonywania się benchmarków wprowadzono górną granicę czasu na każdy test case: 300 sekund.

\subsection{Zestaw 1}

Jak widać na wykresach załączonych poniżej według oczekiwań największy wpływ na czas wykonywania/używaną pamięć największy wpływ ma ilość klauzul.


%\begin{figure}[H]
%  \centerline{\includegraphics[width=0.9\textwidth]{outputs/set1/set1 charts/01 Prover9 number of clauses vs time.jpg}}
%  \caption{Prover9 zestaw 1 liczba klauzul vs czas}
%\end{figure}


\subsection{Zestaw 2}

W zestawie 2 badano wpływ stosunku ilości atomów bezpieczeństwa do ilości atomów żywotności. W grupach test case'ów o tych samych ilościach klauzul zadano różne stosunki ilości atomów bezpieczeństwa do ilości atomów żywotności. 

\subsection{Zestaw 3}

W zestawie 3 badano wpływ udziału klauzul k-SAT na czas wykonywania/pamięć w zależności od k. Zadano klauzule o różnym stopniu rozłożenia spośród wartości k $\epsilon$ \{1,5,10,20\}. 

\printglossary

\end{document}
